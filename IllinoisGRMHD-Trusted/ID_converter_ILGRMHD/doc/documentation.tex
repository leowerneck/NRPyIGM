\documentclass{article}
%\usepackage{../../../../doc/ThornGuide/cactus}
\usepackage{../../../../doc/latex/cactus}
\begin{document}

% The title of the document (not necessarily the name of the Thorn)
\title{{\tt ID\_Converter\_ILGRMHD}: A Module for Converting {\tt HydroBase} Initial Data into Variables {\tt IllinoisGRMHD} Can Read}

% The author of the documentation - on one line, otherwise it does not work
\author{Original author: Zachariah B. Etienne. }

% the date your document was last changed, if your document is in CVS, 
% please use:
\date{$ $Date: 2015-10-12 12:00:00 -0600 (Mon, 12 Oct 2015) $ $}
\maketitle

% *======================================================================*
%  Cactus Thorn template for ThornGuide documentation
%  Author: Ian Kelley
%  Date: Sun Jun 02, 2002
%  $Header$                                                             
%
%  Thorn documentation in the latex file doc/documentation.tex 
%  will be included in ThornGuides built with the Cactus make system.
%  The scripts employed by the make system automatically include 
%  pages about variables, parameters and scheduling parsed from the 
%  relevent thorn CCL files.
%  
%  This template contains guidelines which help to assure that your     
%  documentation will be correctly added to ThornGuides. More 
%  information is available in the Cactus UsersGuide.
%                                                    
%  Guidelines:
%   - Do not change anything before the line
%       % START CACTUS THORNGUIDE",
%     except for filling in the title, author, date etc. fields.
%        - Each of these fields should only be on ONE line.
%        - Author names should be sparated with a \\ or a comma
%   - You can define your own macros are OK, but they must appear after
%     the START CACTUS THORNGUIDE line, and do not redefine standard 
%     latex commands.
%   - To avoid name clashes with other thorns, 'labels', 'citations', 
%     'references', and 'image' names should conform to the following 
%     convention:          
%       ARRANGEMENT_THORN_LABEL
%     For example, an image wave.eps in the arrangement CactusWave and 
%     thorn WaveToyC should be renamed to CactusWave_WaveToyC_wave.eps
%   - Graphics should only be included using the graphix package. 
%     More specifically, with the "includegraphics" command. Do
%     not specify any graphic file extensions in your .tex file. This 
%     will allow us (later) to create a PDF version of the ThornGuide
%     via pdflatex. |
%   - References should be included with the latex "bibitem" command. 
%   - use \begin{abstract}...\end{abstract} instead of \abstract{...}
%   - For the benefit of our Perl scripts, and for future extensions, 
%     please use simple latex.     
%
% *======================================================================* 
% 
% Example of including a graphic image:
%    \begin{figure}[ht]
%       \begin{center}
%          \includegraphics[width=6cm]{MyArrangement_MyThorn_MyFigure}
%       \end{center}
%       \caption{Illustration of this and that}
%       \label{MyArrangement_MyThorn_MyLabel}
%    \end{figure}
%
% Example of using a label:
%   \label{MyArrangement_MyThorn_MyLabel}
%
% Example of a citation:
%    \cite{MyArrangement_MyThorn_Author99}
%
% Example of including a reference
%   \bibitem{MyArrangement_MyThorn_Author99}
%   {J. Author, {\em The Title of the Book, Journal, or periodical}, 1 (1999), 
%   1--16. {\tt http://www.nowhere.com/}}
%
% *======================================================================* 

% If you are using CVS use this line to give version information
% $Header$

% Use the Cactus ThornGuide style file
% (Automatically used from Cactus distribution, if you have a 
%  thorn without the Cactus Flesh download this from the Cactus
%  homepage at www.cactuscode.org)

% Do not delete next line
% START CACTUS THORNGUIDE

% Add all definitions used in this documentation here 
%   \def\mydef etc

%\newcommand{\eqref}[1]{(\ref{#1})}

% Add an abstract for this thorn's documentation
\begin{abstract}
{\tt IllinoisGRMHD} and {\tt HydroBase} variables are incompatible;
The former uses 3-velocity defined as $v^i = u^i/u^0$, and
the latter uses the Valencia formalism definition of $v^i$.

Define the Valencia formalism's definition of $v^i$ to be 
"$W^i$", and {\tt IllinoisGRMHD}'s definition "$v^i$"
Then
\begin{equation}
W^i = (v^i + \beta^i) / (\alpha),
\end{equation}
which comes from Eq 11 in the {\tt IllinoisGRMHD} code announcement
paper:\\ \url{http://arxiv.org/pdf/1501.07276.pdf}.

Similarly,
\begin{equation}
v_i = (\alpha) W^i  - \beta^i
\end{equation}

In addition, {\tt IllinoisGRMHD} needs the A-fields to be 
defined on {\it staggered} grids, and {\tt HydroBase} does not yet
support this option. The staggerings are defined in 
Table 1 of the {\tt IllinoisGRMHD} code announcement paper:
\\ \url{http://arxiv.org/pdf/1501.07276.pdf} (page 15).

The long-term goal should be to adjust {\tt HydroBase} and {\tt
  IllinoisGRMHD} so that this thorn is no longer necessary.
\end{abstract}

% Do not delete next line
% END CACTUS THORNGUIDE

\end{document}
